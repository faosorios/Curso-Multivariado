\documentclass[11pt]{exam}
\usepackage[utf8]{inputenc}

\usepackage[margin=1in]{geometry}
\usepackage{amsmath,amsfonts,amssymb,amsthm}
\usepackage{mathtools}
\usepackage{multicol}
\usepackage{color,graphics,subfigure}
\usepackage{listings}
\usepackage[nolinks]{qrcode} % create QR codes
\usepackage{hyperref,url}
\hypersetup{colorlinks, citecolor=blue}

\lstset{basicstyle=\ttfamily,
  showstringspaces=false,
  commentstyle=\color{red},
  keywordstyle=\color{blue}
}

\newcommand{\class}{MAT-269: Análisis Estadístico Multivariado}
\newcommand{\term}{Semestre 1, 2023}
\newcommand{\examnum}{Certamen 2. Mayo 17, 2023}
\newcommand{\examdate}{17/05/2023}
\newcommand{\timelimit}{Mayo 17, 2023}

\pagestyle{head}
\firstpageheader{}{}{}
\runningheader{MAT-269}{Certamen 2\ - P\'agina \thepage\ de \numpages}{\examdate}
\runningheadrule

% misc
\makeatletter
\def\tagform@#1{\maketag@@@{\bf(\ignorespaces#1\unskip\@@italiccorr)}}
\makeatother

% Math commands
\DeclareMathOperator{\argmax}{arg\,max}
\DeclareMathOperator{\argmin}{arg\,min}
\DeclareMathOperator{\blkdiag}{blc\,diag}
\DeclareMathOperator{\cor}{cor}
\DeclareMathOperator{\cov}{cov}
\DeclareMathOperator{\D}{D}
\let \d \relax
\DeclareMathOperator{\d}{d}
\DeclareMathOperator{\diag}{diag}
\DeclareMathOperator{\diff}{d}
\DeclareMathOperator{\E}{E}
\DeclareMathOperator{\med}{me}
\let \P \relax
\DeclareMathOperator{\P}{P}
\DeclareMathOperator{\rd}{d}
\DeclareMathOperator{\rk}{rg}
\DeclareMathOperator{\tr}{tr}
\DeclareMathOperator{\var}{var}
\let \vec \relax
\DeclareMathOperator{\vec}{vec}
\def\half{{\textstyle\frac{1}{2}}}
\def\Rset{\mathbb{R}}
\newcommand{\what}[1]{\widehat{#1}}
\newcommand{\bt}[1]{\ensuremath{\mathbf{#1}}} % bold roman
\newcommand{\bm}[1]{\mbox{\boldmath $#1$}}    % bold math mode

\renewcommand{\figurename}{Figura}

\begin{document}

\noindent
\begin{tabular*}{\textwidth}{l @{\extracolsep{\fill}} r @{\extracolsep{6pt}} l}
\textbf{\class} 	& \textbf{Nombre:}	& \makebox[2in]{\hrulefill} \\[.4ex]
\textbf{\examnum} & \textbf{Profesor:} & Felipe Osorio \\[.4ex]
\end{tabular*} \\[.25ex]

\noindent
\rule[2ex]{\textwidth}{1pt}

%-------------------------------------------------------------------------------
\begin{enumerate}

\item[\bf 1.] Sea $(X_1,Y_1)^\top,\dots,(X_n,Y_n)^\top$ una muestra aleatoria 
desde $\mathsf{N}_2(\bm{\mu},\bm{\Sigma})$, con 
\[
  \bm{\mu} = \begin{pmatrix}
    \mu_1 \\
    \mu_2 
  \end{pmatrix}, \qquad \bm{\Sigma} = \begin{pmatrix}
    \sigma_{11} & \sigma_{12} \\
    \sigma_{21} & \sigma_{22}
  \end{pmatrix}.
\]
\begin{itemize}
  \item[\bf a)] Obtenga los MLE de 
  \begin{align*}
    \rho_c & = \frac{2\sigma_{12}}{\sigma_{11} + \sigma_{22} + (\mu_1 - \mu_2)^2}, \qquad \textrm{y} \\
    \Psi_c & = \P(|X_i - Y_i| \leq c) = \Phi\Big(\frac{c - \delta}{\omega}\Big) 
    - \Phi\Big(\frac{-c - \delta}{\omega}\Big),
  \end{align*}
  con $\delta = \mu_1 - \mu_2$, $\omega^2 = \sigma_{11} + \sigma_{22} - 2\sigma_{12}$.

  \medskip 

  \item[\bf b)] Determine la distribución asintótica de $\what{\rho}_c$.
  
  \medskip 

  \emph{Puede ser útil:} Suponga $\bm{T}_n$ un estimador de la forma $\bm{T}_n = \bm{h}(\bm{S}_n)$ 
  donde la secuencia $\{\bm{S}_n\}$ es asintóticamente normal, esto es,
  \[
    \sqrt{n}(\bm{S}_n - \bm{\mu}) \stackrel{\sf D}{\to} \mathsf{N}(\bm{0},\bm{\Sigma}),
  \]
  para $\bm{\mu}\in\Rset^k$ y $\bm{\Sigma} > 0$. Si $\partial\bm{h}(\bm{\mu})/\partial
  \bm{\mu}^\top$ es matriz de rango completo, entonces
  \[
    \sqrt{n}(\bm{T}_n - \bm{h}(\bm{\mu})) \stackrel{\sf D}{\to} \mathsf{N}\Big(\bm{0},
    \Big(\frac{\partial\bm{h}(\bm{\mu})}{\partial\bm{\mu}^\top}\Big)\bm{\Sigma}\,
    \Big(\frac{\partial\bm{h}(\bm{\mu})}{\partial\bm{\mu}^\top}\Big)^\top\Big).
  \]
\end{itemize}

\bigskip

\item[\bf 2.] Suponga $\bm{x}_1,\dots,\bm{x}_n$ una muestra aleatoria desde $\mathsf{N}_p(\bm{\mu},
  \bm{\Sigma})$ con $\bm{\Sigma}$ definida positiva. Determine el LRT para probar $H_0:\bm{\mu} 
  = \bm{\mu}_0, \bm{\Sigma} = \bm{\Sigma}_0$, con $\bm{\mu}_0$ y $\bm{\Sigma}_0$ especificados.

\end{enumerate}

\end{document}
